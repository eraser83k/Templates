\documentclass[../main.tex]{subfiles}
\begin{document}

\vfill
\chapter{Prologue}

\vfill
\newpage
\section{Why Tensor Calculus?}
\subsection{Exercise 1.}
Suppose that the temperature field $T$ is given by the function $F(x,y)=x^2e^y$ in coordinates $x, y$. Determine the function $F(x',y')$, which gives the temperature field $T$ in coordinates $x',y'$.

\sol

$F'(x',y')=F(2x',2y')=F(x,y)=(2x')^2e^{2y'}\QED$

\subsection{Exercise 2.}

This is a table.

\begin{table}[h]
    \centering
    \caption{Student Academic Performance - Fall 2024}
    \label{tab:student_performance}
    \begin{tabular}{@{}lcccr@{}}
        \toprule
        Student Name & Mathematics & Physics & Chemistry & GPA \\
        \midrule
        John Smith & 85 & 92 & 78 & 3.52 \\
        Emily Johnson & 94 & 88 & 91 & 3.78 \\
        Michael Brown & 76 & 82 & 85 & 3.21 \\
        Sarah Davis & 91 & 95 & 89 & 3.85 \\
        David Wilson & 88 & 79 & 84 & 3.44 \\
        \midrule
        Average & 86.8 & 87.2 & 85.4 & 3.56 \\
        \bottomrule
    \end{tabular}
\end{table}

\subsection{Exercise 3.}
The derivation of the Black-Scholes equation involves the use of Ito's Lemma and the concept of a risk-neutral portfolio. Consider a stock whose price \( S(t) \) follows the stochastic differential equation:

\begin{equation}
    dS = \mu S dt + \sigma S dW
\end{equation}

where:
\begin{itemize}
    \item \( \mu \) is the drift rate of the stock.
    \item \( \sigma \) is the volatility of the stock.
    \item \( W \) is a Wiener process or Brownian motion.
\end{itemize}

\defn{The Formula}{
    \begin{equation}
    dS = \mu S dt + \sigma S dW
\end{equation}
}

\end{document}