\documentclass[../main.tex]{subfiles}
\begin{document}

\section{Exercise 1.}
Suppose that the temperature field $T$ is given by the function $F(x,y)=x^2e^y$ in coordinates $x, y$. Determine the function $F(x',y')$, which gives the temperature field $T$ in coordinates $x',y'$.

\sol

$F'(x',y')=F(2x',2y')=F(x,y)=(2x')^2e^{2y'}\QED$

\section{Exercise 2.}

This is a table.

\begin{table}
\caption{Enter table caption here.}
\begin{center}

\begin{tabular}{@{}cccc@{}}
\toprule
Tap     &Relative   &Relative   &Relative mean\\
number  &power (dB) &delay (ns) &power (dB)\\
\midrule
3 &0$-9.0$  &68,900\footnotemark[1] &$-12.8$\\
4 &$-10.0$ &12,900\footnotemark[2] &$-10.0$\\
5 &$-15.0$ &17,100 &$-25.2$\\
\bottomrule
\end{tabular}
\end{center}
\end{table}

\section{Exercise 3.}
The derivation of the Black-Scholes equation involves the use of Ito's Lemma and the concept of a risk-neutral portfolio. Consider a stock whose price \( S(t) \) follows the stochastic differential equation:

\begin{equation}
    dS = \mu S dt + \sigma S dW
\end{equation}

where:
\begin{itemize}
    \item \( \mu \) is the drift rate of the stock.
    \item \( \sigma \) is the volatility of the stock.
    \item \( W \) is a Wiener process or Brownian motion.
\end{itemize}

\defn{The Formula}{
    \begin{equation}
    dS = \mu S dt + \sigma S dW
\end{equation}
}

\end{document}